%%%%%%%%%%%%%%%%%%%%%%%%%%%%%%%%%%%%%%%%%
% Medium Length Professional CV
% LaTeX Template
% Version 2.0 (8/5/13)
%
% This template has been downloaded from:
% http://www.LaTeXTemplates.com
%
% Original author:
% Trey Hunner (http://www.treyhunner.com/)
%
% Important note:
% This template requires the resume.cls file to be in the same directory as the
% .tex file. The resume.cls file provides the resume style used for structuring the
% document.
%
%%%%%%%%%%%%%%%%%%%%%%%%%%%%%%%%%%%%%%%%%

%----------------------------------------------------------------------------------------
%	PACKAGES AND OTHER DOCUMENT CONFIGURATIONS
%----------------------------------------------------------------------------------------


\documentclass{resume} % Use the custom resume.cls style
\nonstopmode

\usepackage[left=0.75in,top=0.6in,right=0.75in,bottom=0.6in]{geometry} % Document margins

\name{Vishnu Nittoor} % Your name
\address{+91 7975322206 \\ nitvishn.github.io \\ nitvishn@gmail.com} % Your phone number and email

\begin{document}

%----------------------------------------------------------------------------------------
%	EDUCATION SECTION
%----------------------------------------------------------------------------------------

\begin{rSection}{Education}

{\bf University of Toronto} \hfill {\em 2021-2025} \\ 
Honours Bachelor of Science, incoming freshman \\
\textbf{Specialist} in Computer Science and \textbf{Major} in Mathematics \smallskip \\
Expected graduation: 2025  \\

{\bf The International School Bangalore} \hfill {\em 2015-2021} \\ 
International Baccalaureate Diploma, \textbf{Score: 44/45}  \\
International General Certificate of Secondary Education, \textbf{8A*, 1A}\\

\end{rSection}

%----------------------------------------------------------------------------------------
%	RESEARCH EXPERIENCE SECTION
%----------------------------------------------------------------------------------------

\begin{rSection}{Research Experience}



\begin{rSubsection}{ A Monte-Carlo analysis of competitive balance and reliability across tournament structures }{ July 2020 - September 2020 }{}
\item  This summer, I had the privilege of working with Prof. Tim Chartier as part of the Pioneer Research Program in the mathematics of ranking. After a 10-week program learning ranking methods, I produced a highly theoretical research paper on a Monte-Carlo analysis of competitive balance and reliability across tournament structures. The paper has impacted Prof Chartier's independently conducted work and opens up several additional questions regarding ranking and tournament structures.  

\end{rSubsection}



\begin{rSubsection}{ On detecting face cycles in incrementally constructed planar graphs }{ October 2019 - March 2019 }{}
\item While programming "Pentalink", a turn-based strategy game, I had to solve a computational geometry problem using an algorithm found in "Polygon Detection from a Set of Lines" by A. Ferreira et al. from the Technical University of Lisbon as a reference. Intrigued, I independently began to research the problem further. After realizing that I can approach this using a wireframe solution from solid modelling, I conducted research comparing the computational complexity of the two approaches under Dr Sashikumar, whom I met during my internship, now a research scientist at Google. It newly considers the incremental addition of edges, as is the case in Pentalink. Conducting deeper research into one of my own projects was equally challenging as it was enriching. 

\end{rSubsection}


\end{rSection}



\begin{rSection}{Work Experience}


\begin{rSubsection}{ Sigmoid Labs (now acquired by Google) }{ June 1 2018 - July 20 2018 }{}
\item I interned at Sigmoid Labs (now acquired by Google) in the summer of 2018. Their product, whereismytrain, is a popular Android app used by tens of millions of people in India. While working there, my colleagues had an unsolved problem related to train crossing detection on their hands. \n Interested in solving it, I worked closely with the CEO, Dr Sashikumar Venkatraman and another engineer in developing an iterative algorithm of my own. I also developed an accelerated visual animation of trains’ locations to aid research. After refining my algorithm multiple times, I generated data that made significant progress on a solution which was widely appreciated. I look forward to working in an intensely collaborative and creative environment like this again.

\end{rSubsection}


\end{rSection}


\begin{rSection}{Projects}


\begin{rSubsection}{ Pentalink }{ September 2019 - December 2019 }{}
\item I worked with and led Pentalink's creative team - I was the sole game programmer on this project. Its inception was the product of bored experimentation during a free lesson, and has now evolved into an epic strategic battle of minds, ubiquitously played amongst our close circle of friends during lunch breaks and bus journeys. Using multiple game design concepts such as state stacks, 2D maps, procedural level generation, and GUI design, I programmed the game from scratch in the Lua programming language using LÖVE2D as a game engine and framework. A notable problem I encountered was the detection of face cycles in planar graphs, a topic I researched deeply for the project.

\end{rSubsection}



\begin{rSubsection}{ Project Euler }{ January 2017 - None }{}
\item Solved over 109 problems on Project Euler, programming my own solutions and creating a Python3 library for computatational number theory algorithms. Researched topics in mathematics and computer science in the process. Made all solutions as well as the library open-source on my GitHub page - the library is named 'addmath.py', reminiscently after the IGCSE Additional Mathematics course I wrote exams for in Grade 10. I also earned the top 1% badge on the platform. Members of the website who solve more than 109 problems fall in the top 1% of problem solvers, earning the award as a badge. The answers are integer solutions to mathematics problems which require skills in programming and algorithms to solve. I also created a Discord bot that serves as a wrapper for the library for access within text channels on Discord servers.

\end{rSubsection}



\begin{rSubsection}{ ECHO (Experimental Counterfeiter of Human Occupancy) }{ September 2020 - None }{}
\item <p>\n            This quarantine, I built my own computer. I had spent months researching a specific graphics processing unit that I wanted – a raytracing card with tensor cores. Sure, I wanted it to play video games at ultra-settings, to run research simulations, and to edit videos. There was, however, another crucial reason I was excited: I could now take my online class shenanigans to the next level.<br/><br/>\n            The day I got it, I launched Open Broadcasting Software and got to work. I started rendering videos of myself sitting in my chair, shifting around, revolving, staring at the ceiling, and perhaps taking a little scribble here and there. Loop it back. Render another. Then, the phrases: I'd record multiple variations of myself going "Yes, I understand", “Absolutely”, “No, I don’t think I have any doubts for now”. I recorded myself in tens of different ponderous gazes, understanding nods, and reflective head movements of clear affirmation. “That’s right; I’ll get to work on that right away”.<br/><br/>\n            Soon, I had an entire video library of all possible generic reactions and idle resting positions. Next, I programmed a script that automatically logged into my classes on time.<br/><br/>\n            I named it the Experimental Counterfeiter of Human Occupancy (or ECHO for short). With ECHO, I was finally ready to automate my presence in online class.<br/><br/>\n            I had never looked forward to an online Monday morning this much. I left ECHO running overnight and sat down at my system the next morning. I watched it login into my morning class at 7: 45, greet the class entirely out of context, and proceed to simulate an unnaturally diligent Vishnu staring with focus at his desktop monitor.<br/><br/>\n            It didn’t take me long to realize, sitting in my chair with my celebratory bowl of cereal, that I was being counterproductive. Every time I was prompted to say something, I had to assist ECHO in choosing the correct responses.<br/><br/>\n            Without making novel discoveries in the science of computational intelligence, the road pretty much ended here. Yes, it worked, but it was hilariously useless. Sorry ECHO – if you’re actually sentient and you’re reading this, I’m immensely proud of you. Whether you do it well or not, I’m sure you (or a hyper-evolved version) will continue fulfilling your raison d'etre of pretending to be Vishnu for generations to follow.\n\n        </p>

\end{rSubsection}



\begin{rSubsection}{ Can I Have Water? }{ January 2020 -  }{}
\item In Bangalore, a large part of the water supply network uses tankers for transportation. There are a few problems with this system. Suppliers don’t have accurate data for their customers’ water needs, there is a rampant inequality in water distribution, and the tankers don't have a system to decide routes. After conducting researching the water consumption needs of households within Bangalore, we built a model that uses sinusoidal regression to predict future water consumption. We identified redundant resources within tanker networks and optimised routes, resulting in a fewer number of tankers needed to be used, and a higher number of communities fulfilled per tanker. We also built a client side interface for households and communities to register themselves and track their water usage. The algorithms that ran in the backend are on the linked repository.

\end{rSubsection}



\begin{rSubsection}{ GreenWheels }{ January 2018 -  }{}
\item Designed a carpooling system to be used amongst members of residential communities in Bangalore. I solved the travelling salesman problem using a greedy approach, and built an algorithm that optimises carpooling groups taking into account parameters such as schedule, social preferences, car size, and more.

\end{rSubsection}



\begin{rSubsection}{ Pong, but the paddles move in 2 dimensions }{ August 2019 -  }{}
\item It's local multiplayer pong, but you can use WASD and arrow keys to move each paddle around in two dimensions. They also teleport when they cross boundaries. It's a very simple change to a classic game, but tt's really, really fun. We played it a lot in dorm, and I was most definitely the best. Probably because I spent so many hours iteratively programming the game, testing it, challenging my friends to duels, losing, and then tweaking the code so that I had an edge the next time. Fun. Interesting.

\end{rSubsection}



\begin{rSubsection}{ Connect 4 Bot }{ April 2017 -  }{}
\item I mean, it's a bot that plays connect4. Self-explanatory. This was during a really intense connect4 craze at our school - everyone was playing connect4 on paper notebooks in between and during classes. There was a very clear dominance heirarchy, and although I wasn't terrible, I wasn't at the top. This bot was. Consistently. So good luck :

\end{rSubsection}



\begin{rSubsection}{ Economics Graph Generator }{ June 2020 -  }{}
\item 

\end{rSubsection}



\begin{rSubsection}{ Charged Particle Simulator }{ February 2018 -  }{}
\item A simulation of physical interactions between charged particles. The simulation works relatively well in terms of the forces between the particles, but partly breaks down when they collide. Psst: if you really want to see it in action, move your mouse around and press the 'p' and 'n' buttons to spawn negative and positive particles. Have fun :

\end{rSubsection}



\begin{rSubsection}{ Vigenere and Subtitution Cipher Cracking Tool }{ August 2016 -  }{}
\item A tool that helps crack vigenere and substitution ciphers quickly. 

\end{rSubsection}



\begin{rSubsection}{ FlowerSpirals }{ February 2018 -  }{}
\item A visual tool to illustrate the beautiful mathematics behind the layout of seeds in sunflowers - each dot representing a seed is drawn at a certain angle from the previous one, forming a spiral. This angle is initialised (the first textbox in the webpage and incremented a little each frame (the second textbox in the webpage). Try putting the golden ratio into the the first box and reload with the changes! Note: please use very small numbers for the second box - a good value to end use is 0.00001.

\end{rSubsection}



\begin{rSubsection}{ Fish Genetics }{ April 2017 -  }{}
\item When our class was doing a Genetics and Environmental Science worksheet, I strangely realised that the test was not completely fair. When we were drawing toothpicks of a particular colour from the gene pool, we were increasing the probability of the other genes to be chosen, therefore rendering the experiment not truly random. When we humans sort the toothpicks and choose "randomly", it was actually not that random, partially because of human error, and the reason I stated above. Also, there were only 24 genes in the gene pool, and we made just 4 generations from them. I decided to write some code, and run a computer simulation.\n\n        I made sure that the genes were randomly chosen, and I ran it for 2400 genes, while making 100 generations. I averaged the results over 100 trials. I generated a few figures as a result, revealing some interesting insights.\n        

\end{rSubsection}



\begin{rSubsection}{ Optimal Roommate Allocator }{ December 2018 -  }{}
\item A program that generates sets of roommates based on a preferential selection system. The algorithm used is pretty close to brute force, and I'm sure there's a way to improve it much further.

\end{rSubsection}



\begin{rSubsection}{ hangman }{ January 2018 -  }{}
\item In English class, our teacher let us play Hangman to pass the time - I'm pretty sure we had earned it in some way or another. I was really bad. I wrote this. It's a very cruel way to treat the people you're playing with - a dictionary search is a loser's way to play the game. Either way, I won a lot.

\end{rSubsection}



\begin{rSubsection}{ Two Player Snake }{ March 2017 -  }{}
\item The Snake game, but now with two players. God, I have so many more of these, but I think this one is the most charismatic. It's very buggy - when we used to play this in my school dormitory at night, the person who was longest when the game crashed due to a bug was declared as the winner. No intentional time limit, just an unintended, unforseeen, inevitable in-game failure. WASD and arrow keys. Half your mass gets re-distributed into the board when you collide either into yourself or your opponent. 

\end{rSubsection}



\begin{rSubsection}{ Tournament Engine for JBPL }{ November 2017 -  }{}
\item 

\end{rSubsection}



\begin{rSubsection}{ Pascal's Triangle... but with RGB colours instead of integers? }{ July 2020 -  }{}
\item 

\end{rSubsection}



\begin{rSubsection}{ The Ultimatum Game }{ July 2020 -  }{}
\item 

\end{rSubsection}


\end{rSection}


\begin{rSection}{Technical Strengths}

\begin{tabular}{ @{} >{\bfseries}l @{\hspace{6ex}} l }
Wireframe & Microsoft Powerpoint \\
Creating a prototype & Invision\\
Visual Design & Paint 3D, Adobe Photoshop, Canva \\
Machine Learning &  Python with Numpy and Pandas \\
Website (coded) &  HTML5 and CSS \\
Data Visualisation & Tableau\\
Typesetting Document & Latex\\
Programming & Java and Java Script, Python, HTML and CSS\\

\end{tabular}

\end{rSection}


%----------------------------------------------------------------------------------------
%	EXAMPLE SECTION
%----------------------------------------------------------------------------------------

%	Volunteering Activities and Projects
%----------------------------------------------------------------------------------------
\begin{rSection}{Volunteering Activities}

\begin{rSubsection}{CoderDojo}{2018-Present}{Champion}{Microsoft Store Sydney and Parramatta}
\item I was responsible for mentoring young people with programming. I facilitate their learning activity by demonstrating the content to cover and assisting them in developing their own small projects or chosen activity. I am also for marketing and liaising with stakeholders who are interested in partnering with CoderDojo.   

\end{rSubsection}
%------------------------------------------------

\begin{rSubsection}{Sydney University}{2013-2014}{Student Representative}{Campberdown,NSW}
\item I gathered information from students and communicate their views to the faculty members. I did this to improve the learning experience of current and future students.

\end{rSubsection}
%------------------------------------------------

\begin{rSubsection}{Aiesec Poland}{November 2013- February 2014}{Intern}{Poznan, Poland}
\item I taught English to Polish Kindergartens and assisted the organisation in promoting Global Host Family and diversity to Polish students. 

\end{rSubsection}

\end{rSection}

%	TECHNICAL STRENGTHS SECTION
%----------------------------------------------------------------------------------------

\begin{rSection}{Awards and Recognitions}

\begin{tabular}{ @{} >{\bfseries}l @{\hspace{6ex}} l }
Techfugees Hackathon (2018) &  Our team (Elevate) won Second Place. \\
Sydney University & Advertised Bursary (2018),NSW Health Grant (2017), Robert Maple Brown (2013)\\
Macquarie University (2013) & Overall Excellence in Science\\
\end{tabular}

\end{rSection}

%\begin{rSection}{Section Name}

%Section content\ldots

%\end{rSection}

%----------------------------------------------------------------------------------------

\end{document}